\chapter{Introduction}

Because of the increasing poularity of home automation, the desire to develop applications for controlling your house became more and more important. Calimero is a collection of Java APIs that together form a foundation for building such applications in EIB/KNX installations. Detailed knowledge of the protocol is not required. On the top of that Calimero only requires J2ME enviorments, which enables use on embedded platforms. It is an open source project and was developed by the Institute of Computer Aided Automation of the Technical University of Vienna.

EIBnet/IP allows you to communicate with components of the widely spreaded EIB/KNX standard by tunnelling over IP Networks. The protocol is published in the KNX Handbook  and in the European standard EN 13321-2:2006 (Open Data Communication in Building Automation, Controls and Building Management - Home and Building Electronic Systems - Part 2: KNXnet/IP Communication), available from European standardization organizations. Note: EN 13321-2 is useless without information on the KNX/EIB control network specific data structures (cEMI, DPTs); these should be defined in EN 50090 (Home and Building Electronic Systems). 

\clearpage
Since development of Calimero many great projects are using it:
\begin{itemize}
	\item \textbf{KNX@Home}(\url{http://knxathome.fh-deggendorf.de/})
	\item \textbf{BASys}(\url{http://sourceforge.net/projects/basys/})
	\item \textbf{KNXnet/IP Wireshark dissector}(\url{http://knxnetipdissect.sourceforge.net/})
	\item \textbf{CONECT}(\url{http://sourceforge.net/projects/conect/})
	\item \textbf{EIB Home Server}(\url{http://eibcontrol.sourceforge.net/})
	\item \textbf{LEIBnix}(\url{http://leibnix.sourceforge.net/Wikka/HomePage})
	\item \textbf{Sombrero}(\url{http://grill.github.com/sombrero/})
	\item \textbf{Scalimero}(\url{http://grill.github.com/SCalimero/})
\end{itemize}
\clearpage
