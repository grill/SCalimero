\chapter{DSL}

\section{Examples}
The scalimero DSL is an easy interface for accessing KNX devices and meant to be used in the Scala interpreter. To use it,
\begin{lstlisting}
import tuwien.auto.scalimero.dsl._
\end{lstlisting}
Then create a network and open it using
\begin{lstlisting}
Network("10.0.0.5") open
\end{lstlisting}
The IP address specified here is the KNX router used to access the network. Then, we need to create some devices.
\begin{lstlisting}
val lA = Lamp("1/1/0")
val lB = Lamp("1/1/1")
\end{lstlisting}
Then we can turn the lamps on and off.
\begin{lstlisting}
lA turn on
lA turn off
\end{lstlisting}
More general:
\begin{lstlisting}
lB send true
lB send false
\end{lstlisting}
We can also read from the devices.
\begin{lstlisting}
val b = lA.read
\end{lstlisting}
Note that b is of type \lstinline!Boolean!, it gets converted from KNX data automatically.

You can also subscribe to events:
\begin{lstlisting}
lA.eventSubscribe(on) {
  println("lA has been turned on")
}
\end{lstlisting}

If you want to be notified for every write on the device, subscribe a write callback:
\begin{lstlisting}
lA.writeSubscribe{
  newstatus : Boolean =>
  println("lA status: " + newstatus)
}
\end{lstlisting}
